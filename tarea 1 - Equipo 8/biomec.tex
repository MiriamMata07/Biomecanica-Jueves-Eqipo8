\documentclass{article}
\usepackage[utf8]{inputenc}
\usepackage[spanish]{babel}

\title{Introducción a la Biomecánica}
\date{\today}
\author{Equipo 8}



\begin{document}
\maketitle
\flushleft

\begin{tabular}{r l}
Jesus Alberto Funes Mendoza  \\
Melissa Lizeth Galindo Reye  \\
Ramón Samuel Blanco Ramirez  \\
Aída Mata Moreno  \\
Miriam Itzel Mata Porras  \\
Victor Alan Higuera Vazquez \\


\end{tabular}

\section{Introduccion}

La Biomecánica es una disciplina que estudia y hace análisis físicos de los movimientos del cuerpo humano. El objetivo de la Biomecánica en las actividades deportivas es la caracterización y la mejora de las técnicas del movimiento a partir de conocimientos científicos. \cite{ff2}

Por esto mismo nosotros como Ingenieros Mecatronicos debemos dominar el tema de la biomecanica ya que en esta se basan muchos de los diseños roboticos y de protesis. Podemos ver un ejemplo en los brazos ensambladores de las lineas automotrices los cuales gracias al estudio de la Biomecanica se les puede o pudo dar un diseño y funcionalidad mas aceptable segun las necesidades.

En este trabajo no solo veremos que es la Biomecanica sino también de donde surge.


\section{Desarrollo}

\subsection{Historia}

Antes de comenzar a conocer el tema, lo primero es conocer la historia de este.

El conocimiento científico tiene su origen en la época griega. Los griegos fueron pioneros en el desarrollo de elementos básicos de matemáticas, física, mecánica o medicina. Durante esta etapa, las principales aportaciones a la biomecánica humana fueron: Separación de conocimiento y mito, varios filósofos griegos aportaron demasiados conocimientos, para para poder aplicarlo. \cite{ff3}


\subsection{Biomedica}

La biomecánica es un área de conocimiento que se interesa por el movimiento, equilibrio, la física, la resistencia, los mecanismos lesionales que pueden producirse en el cuerpo humano como consecuencia de diversas acciones físicas.

La biomecánica se ayuda de otras ciencias como la mecánica y la ingeniería para que, con los conocimientos de anatomía y fisiología del cuerpo humano, poder observar, estudiar y describir el movimiento humano. Existen tres factores por los cuales se deben de estudiar en movimiento la cual es el control, estructura y las fuerzas, estos dos últimos aspectos nos permiten el estudio de los movimientos de los seres vivos desde un punto de vista fundamentalmente anatómico o estructural.\cite{ff4}

La biomecánica estudia a los seres vivos desde el punto de vista de la mecánica, buscando relaciones entre magnitudes y buscando explicaciones de comportamientos y observaciones. Dentro de la mecánica se incluye también todo lo relacionado con los fluidos y la termodinámica. 


\subsection{Disiplinas de la Biomedica}

La biomecánica se considera como la disciplina que estudia los modelos, fenómenos, y leyes que sean relevantes en el movimiento de un ser vivo. Para estudiar el movimiento hay que considerar tres aspectos distintos: 
\begin{enumerate} 
\item El control del movimiento que está relacionado con los ámbitos psicológicos y neurofisiológico.
\item La estructura del cuerpo que se mueve, que en el cas de los seres vivos es un sistema complejo compuesto por músculos, huesos, tendones, etc. 
\item Las fuerzas, tanto externas (gravedad, viento, etc.) como internas (producidas por el propio ser vivo), que producen el movimiento de acuerdo con las leyes de la Física. 
\end{enumerate} 
Los últimos dos aspectos permiten el estudio de los movimientos de los seres vivos desde un punto de vista fundamentalmente anatómico o estructural. Así, los movimientos se deducen sobre todo de la estructura del sistema en movimiento (esqueleto, articulaciones, tendones, músculos, etc.) aplicando tanto las leyes fisiológicas como físicas (mecánicas). \cite{ff5}

\subsection{Estudios de la Biomecanica}

La anatomía es la ciencia que estudia los aspectos estructurales y funcionales del cuerpo. humano, ya sea a nivel macroscópico (donde el objeto de estudio puede ser observado a simple vista) o a nivel microscópico (cuando instrumentos para ampliar lo que se quiere observar). 
El estudio anatómico “[...] es la base de la pirámide a partir de la cual la experiencia del movimiento humano”. \cite{ff6}

\subsection{Anatomia}

 Es esencial conocer la anatomía desde el punto de vista topográfico, siendo necesario aislar las regiones para llevar a cabo las identificaciones morfofuncionales adecuadas huesos, articulaciones, músculos, vasos y nervios. Además, conocer la Principios básicos de la biomecánica son  los aspectos anatómicos que también favorecen la valoración de posibles lesiones. Por ejemplo, suponga que, en rehabilitación, un paciente, un estudiante o un deportista manifiesta una sensación dolorosa en la región interna del codo. En esto región, podemos identificar, a partir de nuestro conocimiento de la anatomía, el epicóndilo medial del húmero como una característica ósea prominente, además de músculos que actúan sobre el carpo y los dedos, permitiendo que ambos la mano y los dedos se dirigen hacia el antebrazo, en un movimiento de flexión Así, el conocimiento de la anatomía puede llevarnos a diagnosticar epicondilitis medial, cuya posible causa fue el amplio uso de los músculos flexores del carpo y de los dedos. 

La anatomía funcional se refiere al área que se ocupa del estudio de las estructuras corporales que se solicitan durante la ejecución de los movimientos humanos Según Hamill: [...] la anatomía funcional, utilizada en el análisis de una elevación lateral del brazo con una mancuerna, identifica el deltoides, el trapecio, el elevador de la escápula, romboides y supraespinoso como contribuyentes a la rotación y elevación hacia arriba del cingulado del miembro superior y para la abducción del brazo. 
Por lo tanto, conocer la anatomía y sus correlaciones funcionales puede ser útil. de varias maneras y en diferentes contextos, como en programas entrenamiento, con o sin el uso de cargas, y en evaluaciones de lesiones en el deporte. El enfoque de la anatomía funcional está en la acción muscular, y no necesariamente en su ubicación. \cite{ff6}

\begin{itemize}
\item Movimiento lineal y movimiento angular
\end{itemize} 

El movimiento ocurre cuando la posición de un cuerpo dado cambia en en relación con el entorno en el que se inserta. En el caso de los movimientos humanos, podemos dividirlos en dos variantes: lineales y angulares.

El movimiento lineal, o movimiento de traslación, ocurre con un proyección rectilínea o curvilínea, de modo que todas las interfaces de objetos que se mueve recorren la misma distancia en el mismo tiempo. Son ejemplos de movimiento lineal las trayectorias de un velocista, una pelota béisbol, un press de banca con barra o un pie cuando se mueve a patear una pelota de fútbol. La atención central de estas acciones es precisamente en la definición de la dirección, trayectoria y/o velocidad a la que se mueve el objeto se mueve o es impulsado a moverse. El centro de masa del cuerpo o 14 Principios básicos de la biomecánica de un segmento del cuerpo es un área que también puede ser monitoreada por análisis lineales. Esto significa que es posible identificar el punto de concentración de masa de un cuerpo para conocer el efecto gravitacional ejerció sobre él. Así, cualquier segmento de un cuerpo u objeto puede individualizarse para que se realice un análisis lineal.

El movimiento angular ocurre alrededor de un punto, de modo que diferentes áreas de un segmento común no se mueven en distancia y tiempo es igual Se puede observar un movimiento angular en, por ejemplo, un giro que ocurre alrededor de una barra colgante. Para completar el giro, los pies necesitan moverse una distancia marcadamente mayor que la que recorrerán las extremidades superiores; esto se debe a que los pies están más separados el punto en el que se produce la rotación. En biomecánica, es común analizar una acción primero desde un punto de vista lineal y luego analizarla desde un punto de vista angular, más concretamente. Se considera que los movimientos angulares originan o ayudan a los movimientos lineales y que, por ello, es positivo analizar ambas vías. \cite{ff6}
\begin{itemize}
\item Cinemática y cinética
\end{itemize} 

La cinemática y la cinética representan dos formas de realizar análisis biomecánicos. La cinemática considera las particularidades del movimiento y las analiza en función del espacio y el tiempo, sin calcular las fuerzas que lo causan o que interfieren con él. a través de este tipo análisis, es posible identificar y describir la velocidad con la que un objeto se mueve, qué tan alto alcanza y qué tan lejos se mueve. De esta forma, los puntos que se consideran en un análisis cinemático son la posición, velocidad y aceleración de un objeto.

Como ejemplo de un análisis cinemático lineal, podemos pensar en la descripción de un atleta que realiza un salto de altura, o en el examen de rendimiento de nadadores de élite. El análisis cinemático angular se puede ejemplificado considerando una secuencia de movimientos articulares para un servicio de tenis y la cadencia segmentaria en un salto vertical.

Del análisis cinemático de movimientos angulares y/o lineales, es posible mejorar el patrón de acciones y así lograr el progreso técnico, además de comprender mejor el movimiento humano. Cuando una mesa es empujada, por ejemplo, puede moverse o no, dependiendo de la fuerza del empuje. Principios básicos de la biomecánica: La cinética, a diferencia de la cinemática, examina no solo las fuerzas que provocan el movimiento, pero también las que interfieren con él. En comparación con un análisis cinemático, un análisis cinético se considera más complejo, tanto por su nivel de comprensión como por sus formas de evaluación Y esta mayor complejidad se debe principalmente a fuerzas no son necesariamente visibles. \cite{ff6}
\begin{itemize}
\item Estático y dinámico
\end{itemize} 
La estática examina los sistemas que están estacionarios o en movimiento en velocidad constante. Es un área de la mecánica que considera que Los sistemas estáticos están en equilibrio, es decir, en una condición estable, en el que la aceleración está ausente debido a la oposición de fuerzas opuestas.

A continuación, vea cómo es posible examinar una postura estática. Observa la postura de una persona sentada en una silla para trabajar en una computadora. Suponga que se están ejerciendo fuerzas, a pesar de no estar identificadas sin movimiento.

Las fuerzas se concentran entre la región dorsal del tronco y la silla, así como entre el suelo y los pies. Las fuerzas musculares, a su vez, contrarrestan la acción de la gravedad. en relación con la postura general del cuerpo, además de mantener la cabeza y tronco erguido. Por lo tanto, las fuerzas identificadas en un escenario estático, como en esta situación descrita, se mantienen para estabilizar una postura que no incluye movimientos.  Dinámica, por otro lado, propone examinar los sistemas móviles, que están en aceleración: a través de un enfoque cinemático o cinético. Tomar ventaja de la postura estática que acabamos de usar como ejemplo, podemos empezar a un análisis dinámico desde el momento en que la persona se levanta de la silla para dejar la mesa del ordenador, es decir, cuando empieza a moverse. Entonces, volviendo a la cinemática y la cinética, podemos lanzar mano para analizar en este ejemplo las fuerzas aplicadas al suelo y articulaciones, así como las fuerzas que actúan sobre el movimiento de la persona al levantarse de la silla y alejarse de la mesa. \cite{ff6}
\bibliography{bib}
\bibliographystyle{plainnat}

\end{document}